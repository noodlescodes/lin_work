\documentclass[a4paper]{article}

\usepackage{color}
\usepackage{listings}
\usepackage{amsmath}
\usepackage{amsthm}
\usepackage{tikz}
\usepackage[]{algorithm,algpseudocode}

\usepackage[hmargin=3cm,vmargin=3.5cm]{geometry}

\newtheorem{theorem}{Theorem}
\newtheorem{definition}{Definition}
\newtheorem*{definition*}{Definition}

\begin{document}

\section{Introduction}

\section{Background}

\section{Algorithm}
This section describes an algorithm to calculate the Weiner index of a Hamming graph. The algorithm can be separated into two components:
\begin{enumerate}
\item Find the elementary cuts;
\item Count the number of vertices on either side of all the cuts.
\end{enumerate}

The first of these can be achieved by the following algorithm.

\begin{algorithm}
  \caption{Elementary Cut Finding Algorithm}\label{euclid}
  \begin{algorithmic}[1]
      \For{all edges in $G$}
      	\State Create a \emph{centre} on each adjacent face.
      	\State Create an edge from the centre to the current edge.
      	\State Increment a counter on the centre by one, to keep track of the degree of each centre.
      \EndFor
      \For{all boundary edges}
      	\State Create a \emph{cut} by tracing the unique path from edge to centre to the opposite edge, until another boundary edge is reached.
      \EndFor
  \end{algorithmic}
\end{algorithm}


\end{document}

TODO:

 - Define opposte edge